On the surface, it appears that SSS is a brute-force approach to approximating a PS Map.  This is a reasonable observation, given that it evaluates every problem instance against every discovered solution.  This is particularly apparent when comparing SSS to SVM-based approaches.  The SVM-based approaches are able to assign a solution to a large number of problem instances based upon the computation of their maximal margin plane.  This would seem to be a more efficient way to generate the approximate PS Map, however, the timing results reveal that this is not the case.  Referring to  the complexity of the two algorithms as summarized in Table \ref{tab:summary-of-complexity} shows that SSS has a complexity of the second order, while the SVM-based approaches have complexities of the third order.  Thus, as the problem space increases, SSS will tend to be more efficient.  

Intuitively, we can compare the two algorithms at a high level:  SSS requires an initial sample, storage of the discovered solutions, and then comparison of each problem instance to each discovered solution.  SVM-based approaches require an initial sample, computation and storage of the maximum margin plane, and then comparison of s
