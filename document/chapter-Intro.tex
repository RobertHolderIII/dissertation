\chapter{Introduction \& Motivation}
\thispagestyle{plain}

\label{ch:introduction}

The ability of a planning system to quickly adapt to environmental changes is critical in time-constrained domains.  Online, heuristic plan repair approaches are sufficient for small changes in the environment; however, repeated or large changes can cause plan quality to degrade. I present an approach that uses available offline time to analyze the space of potential changes in the environment and creates a mapping between problem instances and solutions for use during runtime.  I show that this approach allows a system to rapidly adapt to changes, while yielding plan quality that is comparable to traditional online approaches.


The motivation for this work stems from the common theme encountered in the course of my work in several domains including shipboard computing resource management, mine-like object visitation, and mobile sensor scheduling.  The shipboard computing resource management task is to allocate computing resources to critical ship processes to support overall ship function. However, if a negative event impacts the ship, then a significant number of the computing resources can suddenly become unavailable.  In this case, the system must quickly reallocate the remaining computing resources to the critical processes such that overall ship function remains viable.

In the mine-like object (MLO) visitation domain, one or ships must visit all of the MLOs in a region to determine if the object is actually a mine or some other innocuous artifact.  Ideally, this would be done as quickly as possible in order to certify the region as safe for travel.  During the course of visiting the known MLOs, a new MLO may be detected by a satellite or other means.  In this case, the routes of the ships in the region must be recalculated to incorporate this new location.

The mobile sensor scheduling problem is similar to the MLO domain except that, one, the mobile sensor only has to pass close enough to a location to receive of sample of its broadcasts, and, two, it must sample the broadcasts within one or more time windows.  In this problem, similar to the MLO domain, new locations and time windows can be introduced and must be incorporated into the schedule of each sensor.

In all of these domains, it is useful to maintain a high quality plan, even when the requirements of the problem change.  The range of change typically occurs on a small number of dimensions, and arrive slowly enough such that replanning does not have to take into account a large number of changes.  Thus, the experiments in this domain tend to reflect that bias.  However, future work does discuss how the approaches proposed in the dissertation would scale to larger problems.



\section{Planning}

Planning is the branch of artificial intelligence concerned with efficiently generating sequences of actions, i.e., \textit{plans}, to achieve goals.  Typically, a planning domain consists of a set of states, described by state variables; a set of available actions, described by their effects on state variables; and one or more goal states.  A planning problem defines a starting state, and the task of the planner is to find a set of actions that transform the starting state into a goal state.  Depending on the complexity of the problem, finding any feasible plan may be satisfactory; in other cases, finding the least expensive plan in terms of length or some total action cost is desired.

Planning for environments in which the planner has a limited amount of time to produce a plan is called \textit{time-constrained planning}.  This type of planning applies to situations in which the usefulness, or \textit{utility}, of a plan degrades over time.  Typically, a tradeoff exists between spending more time searching the space for a better plan and quickly deciding on a plan that may have lower utility.  When some prior plan already exists, the planner can either repair the current plan, which is typically faster, or replan, which typically yields better utility.


%For example, consider the classic planning domain of Blocks World.  In this domain the planner must produce one or more stacks of blocks at a one or more specific locations from a set of blocks scattered or stacked on a table.  The planner is allowed to move a block to a location on the table or on top of another block.  Thus a state consists of the location of each of the blocks.  Actions might be defined such as ``move-block \textit{block} onto \textit{object}'' with the precondition that the block does not have another object on top of it, or is \textit{clear}.  Additionally, the object must either be the table or a distinct block that is clear.

I will use the traveling salesman problem (TSP) planning problem as a reference problem throughout much of this dissertation.  The TSP requires a solver to find the shortest route that visits a given set of locations.  This is a classic NP-complete problem that has been studied widely in computer science.  In the basic problem, all of the locations are static.  The dynamic variant, the DTSP, allows locations to be introduced to the planner after execution begins.


\section{Overview of Problem Space Approximation}

Instead of computing approximate solutions at runtime, my approach is to precompute a library of high-quality solutions \textit{prior} to runtime.  In the case of DTSP, one could imagine a library containing a high-quality\footnote{``High-quality'' refers to the plan generated by an offline heuristic solver.  Since a heuristic solver creates an approximate solution, the result cannot be assumed to be optimal.  Thus, I describe the resulting solutions as ``high-quality'' rather than optimal, ideal, or exact.} solution for every possible combination of potential new destinations.  Obviously, as the scale of the planning problem increases, the level of complexity precludes creating a comprehensive library, so in practice a library can only contain a subset of all possible plans. Therefore, I also introduce methods to ensure that the library contains appropriate plans for use when the planning environment changes.

An understanding of the problem space characteristics can be used to choose the planning scenarios for which to generate solutions.  In particular, identifying regions of problem instances with identical solutions allows for the efficient creation of a mapping from problem instances to solutions, called a \textit{Problem-Solution Map} (PS Map).  A PS Map is a component of \textit{problem space analysis} (PSA), which allows a system to make informed decisions about which solutions to include in the library.

Problem instances contain characteristics that are identical, called \textit{static characteristics}, and characteristics that differ between them, the \textit{variable features}, that lead to differences in the problem instance solutions.  The PS Map represents a library of solutions for problem instances, indexed by the variable features of the set of problem instances.  This map provides a mapping from a problem instance to its solution, showing the changes in the solutions as a function of the variable features within the problem instances.  I will discuss several techniques to efficiently build this map.

\section{Summary of Contributions}

This dissertation contributes an approach to real-time planning that leverages offline time to generate a plan library.  Chapter \ref{ch:psa} introduces the concept of a Problem-Solution (PS) Map, and describes several novel approximation approaches in order to create the plan library.  I note how the solution spaces of a domain can have homogeneous regions that can be exploited to efficiently find solutions to a large number of problem instances. The most promising algorithms are those that are able to quickly find the borders of the solutions regions.

I then demonstrate this approach's applicability to multiple domains through experiments in Chapter \ref{ch:evaluation}.  These experiments illustrate that good approximate PS Maps can be obtained from a small number of samples in the problem space.  I also demonstrate how creating abstract solutions allows these algorithms to be  utilized in a domain in which the solutions do not form homogeneous regions.

This dissertation also briefly examines practical tradeoffs between online and offline planning time in Chapter \ref{ch:application}.  This includes some timing results and thoughts on choosing a sample rate and the appropriate algorithm. This chapter also revisits the issue of irregular solutions spaces, and discusses reindexing a solution space as a technique to facilitate the use of PS Map approximation algorithms.

Finally, Chapter \ref{ch:future} suggests extensions to this work and concludes the dissertation.



