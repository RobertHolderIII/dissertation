\documentclass[12pt]{report}

\usepackage{multirow}
\usepackage{doublespace}
\usepackage{csmthesis}
\usepackage{aaai-bib}
\usepackage[round,authoryear]{natbib} % for citation style

%Place content-specific packages
\usepackage{times}
\usepackage{helvet}
\usepackage{courier}
%\usepackage{epsfig}
\usepackage{graphicx}
%\usepackage{subfigure}

%removing this because this template defines \comment{} as a command
%\usepackage{verbatim} % for comment environment

%\usepackage{array} % for table m attribute
\usepackage{amsmath} % for split, align in equation env and text in math env
\usepackage{caption}
%\usepackage{xspace}
%\usepackage[leqno]{amsmath}
%\usepackage{amssymb}
%\usepackage{dsfont}

%\bibliographystyle{mlapa}

\usepackage{geometry}
%\geometry{verbose, letterpaper, dvips, tmargin=1in, lmargin=1.6in, rmargin=0.9in, bmargin=1in, total={6in,9in}, includeheadfoot, headsep=12pt}
\geometry{verbose, letterpaper, dvips, tmargin=1in, lmargin=1.6in, rmargin=0.9in, bmargin=1in, total={6in,9in}, headsep=12pt}

\usepackage{algpseudocode}  % for algorithm formatting
\usepackage{algorithm}   % for algorithm formatting

\usepackage{amsthm} %for theorem and proof env command
\usepackage{amsfonts} % for integer set symbol and mathcal
\usepackage{url} %bib file uses \url
\usepackage{array} % for table m attribute
\usepackage{listings} % for lstlisting env

\usepackage{lscape} % for landscape

\newcommand{\etal}{et al.\xspace}
\newcommand{\eg}{e.g.,\xspace}
\newcommand{\comment}[1]{}


\def\newnot#1{\label{#1}}

\def\ni{\noindent}
\newcommand{\header}[1]{\noindent \bf{#1}\rm \\ \ni }
% psfigTeX macros
%
% All software, documentation, and related files in this distribution of
% psfig/tex are Copyright (c) 1987 Trevor J. Darrell
%
% Permission is granted for use and non-profit distribution of psfig/tex 
% providing that this notice be clearly maintained, but the right to
% distribute any portion of psfig/tex for profit or as part of any commercial
% product is specifically reserved for the author.
%
% Psfig/tex version 1.1
%
% file last modified: $Header: /home/group-folders/segmentation/cvsroot//devina/peruse/thesis/psfig.tex,v 1.1.1.1 2002/11/04 17:07:44 devina Exp $
%
\catcode`\@=11\relax
\newwrite\@unused
\def\typeout#1{{\let\protect\string\immediate\write\@unused{#1}}}
\typeout{psfig: version 1.1}
\def\psglobal#1{
\typeout{psfig: including #1 globally}
\immediate\special{ps:plotfile #1 global}}
\def\psfiginit{\typeout{psfiginit}
\immediate\psglobal{/usr/lib/ps/figtex.pro}}
%
% @psdo control structure -- similar to Latex @for.
% I redefined these with different names so that psfig can
% be used with TeX as well as LaTeX, and so that it will not 
% be vunerable to future changes in LaTeX's internal
% control structure,
%
\def\@nnil{\@nil}
\def\@empty{}
\def\@psdonoop#1\@@#2#3{}
\def\@psdo#1:=#2\do#3{\edef\@psdotmp{#2}\ifx\@psdotmp\@empty \else
    \expandafter\@psdoloop#2,\@nil,\@nil\@@#1{#3}\fi}
\def\@psdoloop#1,#2,#3\@@#4#5{\def#4{#1}\ifx #4\@nnil \else
       #5\def#4{#2}\ifx #4\@nnil \else#5\@ipsdoloop #3\@@#4{#5}\fi\fi}
\def\@ipsdoloop#1,#2\@@#3#4{\def#3{#1}\ifx #3\@nnil 
       \let\@nextwhile=\@psdonoop \else
      #4\relax\let\@nextwhile=\@ipsdoloop\fi\@nextwhile#2\@@#3{#4}}
\def\@tpsdo#1:=#2\do#3{\xdef\@psdotmp{#2}\ifx\@psdotmp\@empty \else
    \@tpsdoloop#2\@nil\@nil\@@#1{#3}\fi}
\def\@tpsdoloop#1#2\@@#3#4{\def#3{#1}\ifx #3\@nnil 
       \let\@nextwhile=\@psdonoop \else
      #4\relax\let\@nextwhile=\@tpsdoloop\fi\@nextwhile#2\@@#3{#4}}
% 
%
\def\psdraft{
	\def\@psdraft{0}
	%\typeout{draft level now is \@psdraft \space . }
}
\def\psfull{
	\def\@psdraft{100}
	%\typeout{draft level now is \@psdraft \space . }
}
\psfull
\newif\if@prologfile
\newif\if@postlogfile
%%% These are for the option list.
%%% A specification of the form a = b maps to calling \@p@@sa{b}
\newif\if@bbllx
\newif\if@bblly
\newif\if@bburx
\newif\if@bbury
\newif\if@height
\newif\if@width
\newif\if@rheight
\newif\if@rwidth
\newif\if@clip
\def\@p@@sclip#1{\@cliptrue}
\def\@p@@sfile#1{%\typeout{file is #1}
		   \def\@p@sfile{#1}
}
\def\@p@@sfigure#1{\def\@p@sfile{#1}}
\def\@p@@sbbllx#1{
		%\typeout{bbllx is #1}
		\@bbllxtrue
		\dimen100=#1
		\edef\@p@sbbllx{\number\dimen100}
}
\def\@p@@sbblly#1{
		%\typeout{bblly is #1}
		\@bbllytrue
		\dimen100=#1
		\edef\@p@sbblly{\number\dimen100}
}
\def\@p@@sbburx#1{
		%\typeout{bburx is #1}
		\@bburxtrue
		\dimen100=#1
		\edef\@p@sbburx{\number\dimen100}
}
\def\@p@@sbbury#1{
		%\typeout{bbury is #1}
		\@bburytrue
		\dimen100=#1
		\edef\@p@sbbury{\number\dimen100}
}
\def\@p@@sheight#1{
		\@heighttrue
		\dimen100=#1
   		\edef\@p@sheight{\number\dimen100}
		%\typeout{Height is \@p@sheight}
}
\def\@p@@swidth#1{
		%\typeout{Width is #1}
		\@widthtrue
		\dimen100=#1
		\edef\@p@swidth{\number\dimen100}
}
\def\@p@@srheight#1{
		%\typeout{Reserved height is #1}
		\@rheighttrue
		\dimen100=#1
		\edef\@p@srheight{\number\dimen100}
}
\def\@p@@srwidth#1{
		%\typeout{Reserved width is #1}
		\@rwidthtrue
		\dimen100=#1
		\edef\@p@srwidth{\number\dimen100}
}
\def\@p@@sprolog#1{\@prologfiletrue\def\@prologfileval{#1}}
\def\@p@@spostlog#1{\@postlogfiletrue\def\@postlogfileval{#1}}
\def\@cs@name#1{\csname #1\endcsname}
\def\@setparms#1=#2,{\@cs@name{@p@@s#1}{#2}}
%
% initialize the defaults (size the size of the figure)
%
\def\ps@init@parms{
		\@bbllxfalse \@bbllyfalse
		\@bburxfalse \@bburyfalse
		\@heightfalse \@widthfalse
		\@rheightfalse \@rwidthfalse
		\def\@p@sbbllx{}\def\@p@sbblly{}
		\def\@p@sbburx{}\def\@p@sbbury{}
		\def\@p@sheight{}\def\@p@swidth{}
		\def\@p@srheight{}\def\@p@srwidth{}
		\def\@p@sfile{}
		\def\@p@scost{10}
		\def\@sc{}
		\@prologfilefalse
		\@postlogfilefalse
		\@clipfalse
}
%
% Go through the options setting things up.
%
\def\parse@ps@parms#1{
	 	\@psdo\@psfiga:=#1\do
		   {\expandafter\@setparms\@psfiga,}}
%
% Compute bb height and width
%
\newif\ifno@bb
\newif\ifnot@eof
\newread\ps@stream
\def\bb@missing{
	\typeout{psfig: searching \@p@sfile \space  for bounding box}
	\openin\ps@stream=\@p@sfile
	\no@bbtrue
	\not@eoftrue
	\catcode`\%=12
	\loop
		\read\ps@stream to \line@in
		\global\toks200=\expandafter{\line@in}
		\ifeof\ps@stream \not@eoffalse \fi
		%\typeout{ looking at :: \the\toks200 }
		\@bbtest{\toks200}
		\if@bbmatch\not@eoffalse\expandafter\bb@cull\the\toks200\fi
	\ifnot@eof \repeat
	\catcode`\%=14
}	
\catcode`\%=12
\newif\if@bbmatch
\def\@bbtest#1{\expandafter\@a@\the#1%%BoundingBox:\@bbtest\@a@}
\long\def\@a@#1%%BoundingBox:#2#3\@a@{\ifx\@bbtest#2\@bbmatchfalse\else\@bbmatchtrue\fi}
\long\def\bb@cull#1 #2 #3 #4 #5 {
	\dimen100=#2 bp\edef\@p@sbbllx{\number\dimen100}
	\dimen100=#3 bp\edef\@p@sbblly{\number\dimen100}
	\dimen100=#4 bp\edef\@p@sbburx{\number\dimen100}
	\dimen100=#5 bp\edef\@p@sbbury{\number\dimen100}
	\no@bbfalse
}
\catcode`\%=14
%
\def\compute@bb{
		\no@bbfalse
		\if@bbllx \else \no@bbtrue \fi
		\if@bblly \else \no@bbtrue \fi
		\if@bburx \else \no@bbtrue \fi
		\if@bbury \else \no@bbtrue \fi
		\ifno@bb \bb@missing \fi
		\ifno@bb \typeout{FATAL ERROR: no bb supplied or found}
			\no-bb-error
		\fi
		%
		\count203=\@p@sbburx
		\count204=\@p@sbbury
		\advance\count203 by -\@p@sbbllx
		\advance\count204 by -\@p@sbblly
		\edef\@bbw{\number\count203}
		\edef\@bbh{\number\count204}
		%\typeout{ bbh = \@bbh, bbw = \@bbw }
}
%
% \in@hundreds performs #1 * (#2 / #3) correct to the hundreds,
%	then leaves the result in @result
%
\def\in@hundreds#1#2#3{\count240=#2 \count241=#3
		     \count100=\count240	% 100 is first digit #2/#3
		     \divide\count100 by \count241
		     \count101=\count100
		     \multiply\count101 by \count241
		     \advance\count240 by -\count101
		     \multiply\count240 by 10
		     \count101=\count240	%101 is second digit of #2/#3
		     \divide\count101 by \count241
		     \count102=\count101
		     \multiply\count102 by \count241
		     \advance\count240 by -\count102
		     \multiply\count240 by 10
		     \count102=\count240	% 102 is the third digit
		     \divide\count102 by \count241
		     \count200=#1\count205=0
		     \count201=\count200
			\multiply\count201 by \count100
		 	\advance\count205 by \count201
		     \count201=\count200
			\divide\count201 by 10
			\multiply\count201 by \count101
			\advance\count205 by \count201
			%
		     \count201=\count200
			\divide\count201 by 100
			\multiply\count201 by \count102
			\advance\count205 by \count201
			%
		     \edef\@result{\number\count205}
}
\def\compute@wfromh{
		% computing : width = height * (bbw / bbh)
		\in@hundreds{\@p@sheight}{\@bbw}{\@bbh}
		%\typeout{ \@p@sheight * \@bbw / \@bbh, = \@result }
		\edef\@p@swidth{\@result}
		%\typeout{w from h: width is \@p@swidth}
}
\def\compute@hfromw{
		% computing : height = width * (bbh / bbw)
		\in@hundreds{\@p@swidth}{\@bbh}{\@bbw}
		%\typeout{ \@p@swidth * \@bbh / \@bbw = \@result }
		\edef\@p@sheight{\@result}
		%\typeout{h from w : height is \@p@sheight}
}
\def\compute@handw{
		\if@height 
			\if@width
			\else
				\compute@wfromh
			\fi
		\else 
			\if@width
				\compute@hfromw
			\else
				\edef\@p@sheight{\@bbh}
				\edef\@p@swidth{\@bbw}
			\fi
		\fi
}
\def\compute@resv{
		\if@rheight \else \edef\@p@srheight{\@p@sheight} \fi
		\if@rwidth \else \edef\@p@srwidth{\@p@swidth} \fi
}
%		
% Compute any missing values
\def\compute@sizes{
	\compute@bb
	\compute@handw
	\compute@resv
}
%
% \psfig
% usage : \psfig{file=, height=, width=, bbllx=, bblly=, bburx=, bbury=,
%			rheight=, rwidth=, clip=}
%
% "clip=" is a switch and takes no value, but the `=' must be preset.
\def\psfig#1{\vbox {
	% do a zero width hard space so that a single
	% \psfig in a centering enviornment will behave nicely
	%{\setbox0=\hbox{\ }\ \hskip-\wd0}
	%
	\ps@init@parms
	\parse@ps@parms{#1}
	\compute@sizes
	%
	\ifnum\@p@scost<\@psdraft{
		\typeout{psfig: including \@p@sfile \space }
		%
		\special{ps::[begin] 	\@p@swidth \space \@p@sheight \space
				\@p@sbbllx \space \@p@sbblly \space
				\@p@sbburx \space \@p@sbbury \space
				startTexFig \space }
		\if@clip{
			\typeout{(clip)}
			\special{ps:: \@p@sbbllx \space \@p@sbblly \space
				\@p@sbburx \space \@p@sbbury \space
				doclip \space }
		}\fi
		\if@prologfile
		    \special{ps: plotfile \@prologfileval \space } \fi
		\special{ps: plotfile \@p@sfile \space }
		\if@postlogfile
		    \special{ps: plotfile \@postlogfileval \space } \fi
		\special{ps::[end] endTexFig \space }
		% Create the vbox to reserve the space for the figure
		\vbox to \@p@srheight true sp{
			\hbox to \@p@srwidth true sp{
				\hfil
			}
		\vfil
		}
	}\else{
		% draft figure, just reserve the space and print the
		% path name.
		\vbox to \@p@srheight true sp{
		\vss
			\hbox to \@p@srwidth true sp{
				\hss
				\@p@sfile
				\hss
			}
		\vss
		}
	}\fi
}}
\catcode`\@=12\relax




%%Items repeated throughout the thesis
%\newcommand{\thesistitle}{Rapid Adaptation in Real-time Systems through Solution Space Analysis}
\newcommand{\thesistitle}{Rapid Plan Adaptation Through Offline Analysis of Potential Plan Disruptors}
\newcommand{\fullname}{Robert H. Holder, III}

\renewcommand{\algorithmiccomment}[1]{// #1}

\begin{document}

I tested the algorithms using the traveling saleman problem (TSP), knapsack problem, and an elevator problem.  The TSP and knapsack problems are classic domains in optimization and computer science.  The elevator problem is a challenge domain created for the AAAI International Planning Competition (IPC).  All the problems are NP-complete or NP-hard, and quickly become intractable with complex enough problem instances.


I used the TSP as an initial domain for algorithm validation and development.  I generated problem instances ranging from 5 cities to 100 cities, representing a range of problem complexity.  The algorithms were developed based upon the insights from tests of  each preceeding algorithm, which will be reflected in some of the discussion.  The initial algorithm, sample-classification (SC), solves a random sample of the problem instances and uses classification based on nearest neighbor to assign solutions discovered during the initial sample to each unsolved problem instance.  In the initial experiment, all the solutions of solved problem instances within a static radius of an unsolved problem instance  were polled and the solution with the plurality was assigned to the unsolved problem instance.  Subsequent experiments weighted the solutions by the reciprocal of the distance or the distance squared, this giving more weight to the solutions of problem instances closer to  the unsolved instance.  Another approach uses an expanding radius to consider additional solutions if there is not a plurality.  The results of the SC experiments are in figure X.  

Based on the results of the SC experiments, it became apparent that the larger solution regions tended to be represented in the approximated PS Map, but smaller regions tended to disapppear.  This occured due to the lower probability of the initial random sample choosing a problem instance in that region, resulting in either a particular region not being represented in discovered solutions or that other, more probablistic, solutions being assigned.  One attempt to mitigate this effect was inspired by observing that within high-quality PS Map, it appears that more rapid changes in solutions and smaller solution regions tend to exist near city locations.  The SC+bias algorithm attempts to take advantage of this observation by biasing samples in the regions near cities.  The city radius and bias parameters determine, respectively, the radius of the region around a city to apply the bias and how much to bias the samples.  The results of this experiment are in figure X.  The results do not show a clear means of applying the parameters to achieve consistently good results.

Sampling classification with active learning (SC+AL) is another attempt to allow for smaller solution regions to be approximated effectively.  SC+AL may be considered a generalization of SC+bias in that it allows more concentrated sampling in regions of the problem space in which the classification appears ambiguous.  For example, if two solutions are both strong candidates to be assigned to a specific problem instance, then SC+AL would allow the problem instance to be solved rather than risk assigning a incorrect solution.  Similarly, if there are no strong candidates for a particular problem instance, then SC+AL would allow the problem instance to be solved rather than assign an arbitrary solution to it.  The results of this experiment are shown in figure X.

The solution border estimation algorithm (SBE) considers the mathematical features of the TSP.  It calculates the border by recognizing that the border between any two solutions is represented by equating the distance functions of the two solutions.  Unfortunately, at the time of this experiment, I did not find a Java library that could solve the complex equations that resulted from this technique.  The SBE-trace technique is inspired by SBE, however it finds borders between two solutions by searching the space between two problem instances with known solutions.  Thus, a binary search can be employed.  Assuming that the border between two solutions is continuous, then the remainder of the border can be found by comparing the utility of the two solutions at each problem instance.  The results of SBE-trace are shown in figure X.  However, SBE-trace is only suitable for two dimensional PS Map approximation, and thus I did not emphasize this algorithm in the subsequent experiments.

The support vector machine algorithm (SVM) uses a support vector machine to try to generalize the idea of SBE to multiple dimensions.  Support vector machines calculate a maximum margin plane to separate different classes.  The observations in this application are the sampled problem instances labeled with their solutions.  The results of this approach are in figure X.


One disadvantage of the SVM-based approach is that it can misclassify problem instances.  SVM determines the borders between two solution regions by creating a margin as far as possible between known solutions instances.  This results in a border that is approximately midway between known solutions.  This has been shown to be a good optimization technique in general, however, it does lead to misclassifications when the actual border does not comform to this approximation.  By applying additional samples in key locations, the bounds of the margins calculated by the SVM can be made tighter and thus more consistent with the acutal borders.  In this approach, the first step is an initial set problem instances are sampled and solved. The second step applies the binary search used in the SBE-trace algorithm to each distinct pair of solutions.  This results in problem instances that represent solutions that are on the border between the distinct pair of solutions.  Finally, those problem instances and the labeled solutions are added to the training set for the SVM. The results of this approach are in figure X. 



The knapsack domain demonstrates the applicability of the algorithms in a different domain.  On difference between this domain and the TSP domain is that it entails a more abstract representation of distance, as an item's  and weight and value characteristics to not directly correspond to location and  distance as directly as the do the cities within the TSP domain. The high-quality solution PS Map's characteristics also differ in this domain.  For example, looking at the high-quality PS Map, one can see that, whereas the TSP domain had very circular homogeneous regions, the knapsack domain has rectangular homogeneous regions.  I apply the same algorithms to this domain, with the exception of the SBE-trace algorithm which is only suitable for problem spaces of two dimensions.  I expect that performance of the algorithms could be worse in this domain, due to the greater number of solutions and smaller solution region size.


Finally, the elevator domain represents a more traditional planning domain.  The TSP and knapsack domains can be considered optimization problems as well as planning problems.  The elevator domain  falls into the more traditional realm of planning where one has to find steps to accomplish a goal in which there is no direct mathematical representation of the domain.  Also, this domain is expected to be more challenging for the algorithms because the homogeneous regions are likely to be smaller and less regular.  Lastly, this domain represents a another level of abstraction in that the solutions that are applied to the domain are not necessarily those that the algorithms will operate upon.  I also apply the algorithms to this domain, again with the exception of SBE-trace.  I expect that this domain will the most challenging of the three, due to the possibility of changes in optimal plan being very sensitive to changes in the problem instance configuration.



\end{document}
