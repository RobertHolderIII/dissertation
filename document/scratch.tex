
The results of approximating an eight-dimensional PS Map representing a knapsack problem with four variable items are in Figure \ref{?}.  




SSS represents the results of an ideal classification, assuming all of the solutions are discovered during the initial sample.  SVM performs better than SVM+SBE in this domain.  This seems to suggest that in this domain it is more useful to devote samples to exploring the problem instance space rather than narrowing the possible range of solution boundaries.

The results of approximating a four-dimensional PS Map representing a knapsack problem with two variable items are in Figure \ref{?}.  The dottted line represents the utility loss of an online solver using a greedy approach of choosing items in order of their value-to-weight ratio.  The figure illustrates that the SVM algorithm performs better than online repair if when the sample rate is greater than .0008.  The SSS algorithm continues to perform well when the sample rate is high enough to such that the initial sampling discovers the solutions of a large portion of the problem instances in the space.




