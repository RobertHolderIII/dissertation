\chapter{Problem Space Analysis}
\thispagestyle{plain}

\label{ch:psa-utilization}

\begin{figure}
\centering
\includegraphics[scale=0.70]{pics/ladybug-cities-solutions.eps}
\caption{Problem-Solution Map for a 5-city TSP.  Hollow circles represent the locations of the four static city locations, and the axes represent the $x$ and $y$ coordinates of possible locations of the fifth city.  The map shows eight unique high-quality solutions for all possible problem instances.}
\label{fig:ps-map-ladybug-marked-cities}
\end{figure}

\begin{figure}
\centering
\includegraphics[scale=0.16]{pics/knapsack-400-ideal-with-labels.eps}
\caption{Problem-Solution Map for knapsack problem.  Axes represent the possible weight and value characteristics of one additional item that the planner may add to the knapsack.  The map shows eleven unique high-quality solution for all possible problem instances.}
\label{fig:ps-map-knapsack}
\end{figure}

A graphical rendering of a PS Map is shown in Figure \ref{fig:ps-map-ladybug-marked-cities}.  This map shows the solutions for a set of small Traveling Salesman Problems (TSPs). In this case, the static characteristics are the x- and y-coordinates of four destinations that are common to all the problem instances, plus the location of the start of the path (at the central solid circle).  The coordinates of the destinations are (10,10), (20,30), (5,35), and (35,25). The variable features of the problem instances are the $x$ and $y$ coordinates of a fifth destination.  The ranges of these two features, the $x$ and $y$ coordinates of the fifth destination, are represented by the  axes of the PS Map.  At each location in the map, representing the $x$ and $y$ coordinates of the fifth destination, the shortest route is generated as the solution.  Finally, each unique solution, consisting of a sequence of city identifiers, is assigned a color and plotted.  For example, (20,10) represents a problem instance in which the fifth city is located at (20,10), and has a shortest path solution of 0-5-1-3-2-4.  Proceeding in this fashion results in a mapping between each problem instance and the solution representing the shortest route.

As another example, a PS Map for a set of 0-1 Knapsack Problems is depicted in Figure \ref{fig:ps-map-knapsack}.  In this case, the problem instances' static set of characteristics is a set of twenty-two items, each with a weight and value.  The problem instances have two variable features, consisting of the weight and value of an additional item, which  are used as the axes of the PS Map.  The solution is the set of items that maximizes the total value of the knapsack, while constraining the total weight to less than a fixed quantity.

The dimensions of the PS Maps are represented as ordinal domains, which requires the ability to enumerate the values of each dimensions.  Planning problems containing dimensions with discrete domains must define an ordering of the values and nearness metric that defines how ``close'' any two values are.  For example, a ``color'' dimension with domain \{red, green, blue\} must define a strict ordering and nearness metric in order to be used by the algorithms described here.  Dimesions consisting or real values must define a granularity to be used within the algorithms.

There is also in implicit assumption that the domains adhere to the ``real world assumption'', that is, that similar problem instances have similar solutions.  This has the effect of problem instances with similar solutions appearing in homogeneous groups within the PS Map.  This could be seen an analogous an SVM assuming its training samples are linearly separable.  In the case that a domain does not adhere to the real world assumption, there are methods, analogous to the SVM kernel trick, that may allow for my algorithms to be applied.  These ideas are discussed in future work.
